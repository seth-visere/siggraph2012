\documentclass{acmsiggraph}                     % final
%\documentclass[annualconference]{acmsiggraph}  % final (annual conference)
%\documentclass[review]{acmsiggraph}            % review
%\documentclass[widereview]{acmsiggraph}        % wide-spaced review
%\documentclass[preprint]{acmsiggraph}          % preprint

%% Uncomment one of the five lines above depending on where your paper is
%% in the conference process. ``review'' and ``widereview'' are for review
%% submission, ``preprint'' is for pre-publication, and ``final'' is for
%% the version to be printed. The ``final'' variant will accept the 
%% ``annualconference'' parameter, which changes the height of the space
%% left clear for the ACM copyright information.

%% The 'helvet' and 'times' packages define the typefaces used for
%% serif and sans serif type in this document. Computer Modern Roman 
%% is used for mathematics typesetting. The scale factor is set to .92
%% to bring the sans-serif type in line with the serif type.

\usepackage[scaled=.92]{helvet}
\usepackage{times}
\usepackage{url}

%% The 'graphicx' package allows for the inclusion of EPS figures.

\usepackage{graphicx}

%% use this for zero \parindent and non-zero \parskip, intelligently.

\usepackage{parskip}

%% Optional: the 'caption' package provides a nicer-looking replacement
%% for the standard caption environment. With 'labelfont=bf,'textfont=it',
%% caption labels are bold and caption text is italic.

\usepackage[labelfont=bf,textfont=it]{caption}

%% If you are submitting a paper to the annual conference, please replace 
%% the value ``0'' below with the numeric value of your OnlineID. 
%% If you are not submitting this paper to the annual conference, 
%% you may safely leave it at ``0'' -- it will not be included in the output.

\onlineid{0}

%% Paper title.

\title{Page Curling in WebGL}

%% Author and Affiliation (single author).

%%\author{Roy G. Biv\thanks{e-mail: roy.g.biv@aol.com}\\Allied Widgets Research}

%% Author and Affiliation (multiple authors).

\author{Seth Samuel\thanks{e-mail: seth@visere.com, web: http://myclibe.com} }

%% Keywords that describe your work.

\keywords{webgl}

%%%%%% START OF THE PAPER %%%%%%

\begin{document}

\teaser{
  \includegraphics[width=1.5in]{pageturn.png}
  \caption{WebGL page turning in Chrome 18 for OS X.}
}

%% The ``\maketitle'' command must be the first command after the
%% ``\begin{document}'' command. It prepares and prints the title block.

\maketitle

%% Abstract section.

\begin{abstract}

\end{abstract}

%% ACM Computing Review (CR) categories. 
%% See <http://www.acm.org/class/1998/> for details.
%% The ``\CRcat'' command takes four arguments.

%\begin{CRcatlist}
%  \CRcat{K.6.1}{Management of Computing and Information Systems}%
%{Project and People Management}{Life Cycle};
%  \CRcat{K.7.m}{The Computing Profession}{Miscellaneous}{Ethics}
%\end{CRcatlist}

%% The ``\keywordlist'' command prints out the keywords.
%\keywordlist

\section{Introduction}

%% The ``\copyrightspace'' command must be the first command after the 
%% start of the first section of the body of your paper. It ensures the
%% copyright space is left at the bottom of the first column on the first
%% page of your paper.

 \copyrightspace

Our application implements the page turning algorithm specified by \cite{Lichan-Hong:2004fk} and implemented in OpenGL by \cite{Nuon:uq}. Implementation also required the calculation of normals in the vertex shader to allow for different textures on each side of the transformed mesh.


\section{Implementation}
\subsection{Page Curling}
Hong et al. proposed a algorithm for realistic simulation of a turning page that could be calculated in realtime. Page curling is a combination of deformation process and a rotation process. The deformation takes the page as a plane anchored at the origin, and wraps it around a cone whose center lies on the y-axis. The rotation process then rotates the deformed mesh about the y-axis.

The shape and position of the cone varies in three phases to simulate the severe corner curling that occurs at the start, broader curving across the whole surface in the middle, and finally the wobble of settling onto the resting state. The attributes of the cone are calculated in Javascript then passed to WebGL shaders as uniforms. These attributes use sinusoidal easing functions to create a smoother and more realistic motion.

The vertex shader calculates its projection onto the cone as a function of its two-dimensional position on the page's plane.  This projection is done in a function that takes as arguments the uniform parameters and the u-v coordinates of a point in the plane. As discussed below we calculate this three times for each vertex to determine the normal vector after the curling transformation.

The rotation is a simple 180-degree rotation about the y-axis. Without the deformation process the page would execute its turn in a completely rigid fashion. The combination of the deformation and rotation create a realistic motion that approximates the combination of flexibility and rigidity in paper.


\subsection{Two-Sided Textures}
A single mesh is used for each page, which has a different texture for each side. To determine the correct texture for a given fragment, a normal vector is passed for each vertex.

Singe each page is a parameterized plane, it is simple to calculate the position of a vertex, and the position of a hypothetical vertex a small $\delta$ in either direction. The normal vector is then calculated as the cross product of these latter two vectors difference from the origin vertex. If $v_0$ is our projected vertex, then
\begin{eqnarray}
v_s = project(v_0.s + \delta, v_0.t)\\
v_t = project(v_0.s, v_0.t + \delta)\\
v_{normal} = cross(v_s-v_0, v_t-v_0)
\end{eqnarray}

The normal vector is passed to the fragment shader, which determines if the normal is facing toward or away from the camera and picks the appropriate texture. This determination is made by taking the dot product with a unit vector on the z-axis. If the dot product is negative then the secondary texture is sampled (with u and v coordinates appropriately reversed).
%\subsection{Queueing}
%Pages sit in a stack on either side of the active turning page, and are raised into the turning position as needed.


\section{Conclusion}

WebGL offers a powerful environment for 3D presentation on the web. By implementing algorithms in hardware accelerated shaders, smooth animations can be created for effects that cannot be achieved with traditional web technologies. Hong et al.'s page turning algorithm allowed out site to have realistic page turning in a web environment.

\bibliographystyle{acmsiggraph}
\nocite{*}
\bibliography{template}
\end{document}
